\documentclass[12pt, titlepage]{article}

\usepackage{booktabs}
\usepackage{tabularx}
\usepackage{hyperref}
\hypersetup{
    colorlinks,
    citecolor=black,
    filecolor=black,
    linkcolor=red,
    urlcolor=blue
}
\usepackage[round]{natbib}

\title{SE 3XA3: Test Report\\Node Messenger}

\author{Team \#24, Node Messenger
		\\ Tasin Ahmed - ahmedm31
		\\ Shardool Patel - pates25
		\\ Omar Elemary - elemaryo
}

\date{\today}

\begin{document}

\maketitle

\pagenumbering{roman}
\tableofcontents
\listoftables
\listoffigures

\begin{table}[bp]
\caption{\bf Revision History}
\begin{tabularx}{\textwidth}{p{3cm}p{2cm}X}
\toprule {\bf Date} & {\bf Version} & {\bf Notes}\\
\midrule
Date 1 & 1.0 & Notes\\
Date 2 & 1.1 & Notes\\
\bottomrule
\end{tabularx}
\end{table}

\newpage

\pagenumbering{arabic}

This document ...

\newpage
\section{Functional Requirements Evaluation}
	\subsection{User Inputs}
	We tested all the input fields in our messenger app. Through testing, we determined that the login and signup feature work as intended. The messenger returns an error if a field is empty while logging in or registering. It returns an error if the email entered is invalid, if the passwords do not match, and if the password is less than 6 characters. While logging in, the console returns errors correctly if the email and password entered are incorrect. While adding people to a new conversation, the messenger ensures that an email is entered and that it is valid. The messenger prevents the user from sending a message if the message box is empty. Pressing 'Enter' or clicking the 'Send' button both send messages correctly. By testing these user inputs we are able to ensure that the core functions of our messenger work as intended.
	\subsection{Data Storage}
	Testing was done to ensure that data sent from Node Messenger was being stored correctly in the firebase database. Once a user creates a new account, Node Messenger correctly stores their Name, Email, and Password in the database. After a message a sent by the user, it is sent to our database and stored. The database stores the messages, the time it was sent at, the chat it was sent to, and the user it was sent by. Our database consists of 3 collections: Messages, Users, and Conversations. By testing, we made sure that the information being sent to the database gets stored into their respective collection.
	\subsection{Data Retrieval}
	Node Messenger is able to look through the firebase database, and retrieve the necessary data. This is done when a new user is signing up to the website. The messenger must look through our database and return an error if the user email already exists in our database. A similar procedure occurs during login. The messenger must take the entered email and password, and allow the user to login if the credentials match that on the database. The messenger is able to display the correct messages to the correct chat, display the name of the user it was sent by, and the time it was sent at. When adding a user to a conversation, the messenger again looks through our database and ensures the email address that was entered is valid.

\section{Nonfunctional Requirements Evaluation}
	\subsection{Look and Feel}
	For the look and feel of the messenger, we wanted it to look visually appealing. For the theme of the messenger, we tried analogous, complementary, triad, and many shades of color, and felt our current theme suits our messenger the best. After hosting our messenger on Github Pages, we shared the URL with many of our classmates, family, and friends. Many of them provided us with valuable feedback that helped us to further improve our messenger to suit the needs of the majority of the users. Most of the feedback we got from them was very positive and assured us that our messenger achieved the goal of being user-friendly.  
	\subsection{Usability and Humanity}
	Node Messenger is made to be catered towards everyone. To achieve this, we used one of the main methods of software design – Abstraction. This means if a user has used any messaging apps before, they will have no problem migrating to Node Messenger. All the features in our messenger are self-intuitive and do not take much time to get used to. Node Messenger is currently hosted on Github pages making it accessible to anyone the planet. It is able to run on any platform without problems. However, during testing, we realized that our messenger does not respond well to smaller screen sizes. Our goal for our servers was to be able to hold as many people as possible. Currently, due to the limitations of our database, Node Messenger is able to serve about 50 people at once. 
	\subsection{Performance}
	Node Messenger performs very well under normal use. The tests were performed on a Windows machine with Intel i5, 1.60GHz, and  8 GB RAM. Our performance goals for Node Messenger was for it to load the messenger, send messages, receive messages, and load past messages quickly. We timed each of these procedures, and the results prove to be very fast as we intended. 
	\begin{tabular}{c c}
	\hline 
	\textbf{Description} & \textbf{Time} \\ 
	\hline 
	Messenger Load Time & 1.2 s \\ 
	Message Send Time & 0.36 s \\ 
	Message Receive Time & 0.56 s \\ 
	Past Messages Load Time & 0.33 s \\ 
	\hline 
	\end{tabular} 
	\subsection{Operational and Environmental}
	Node Messenger is able to work on all platforms. However, it performs the best on Windows. Due to the scale of this project and the lack of time, responsiveness was not fully implemented. Meaning Node Messenger has difficulty loading components properly on smaller screens. This does not mean the messenger is unusable on these platforms. All functions of the messenger perform as intended. None of our testers have reported any other errors occurring when attempting to load on platforms other than windows. 
	\subsection{Maintainability and Support}
	All functions in Node Messenger has been tested thoroughly to ensure everything performs as intended. We want to make sure to keep the downtime and maintenance to a minimum. All of the source code is well structured and documented to help us quickly find errors, and fix them. This will also help programmers navigate through our code if they want to add any new features to Node Messenger. 
	\subsection{Security}
	All user information is kept safely in our firebase database. Node Messenger will not save any of the personal information nor will it share them with other parties.
	\subsection{Legal}
	Node Messenger abides by all rules and regulations. The public will have access to the open-source project. If they would like to improve our messenger, they are free to do so.  
	
\section{Comparison to Existing Implementation}	

This section will not be appropriate for every project.

\section{Unit Testing}

\section{Changes Due to Testing}

\section{Automated Testing}
		
\section{Trace to Requirements}
		
\section{Trace to Modules}		

\section{Code Coverage Metrics}

\bibliographystyle{plainnat}

\bibliography{SRS}

\end{document}